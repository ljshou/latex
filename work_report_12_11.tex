%\documentclass[slidestop,compress,mathserif,blue]{beamer}
\documentclass[mathserif]{beamer}
%\usepackage[noindent,nofonts]{ctexcap} 
\usepackage{CJK} 
%\setCJKmainfont[BoldFont=STHeiti]{STXihei} 
%\setCJKmainfont[BoldFont=STZhongsong]{STZhongsong} 
\usepackage{beamerthemeshadow}
\usepackage{beamerthemesplit}
%\usetheme{shadow}
\usecolortheme{default}
\setbeamertemplate{footline}[frame number]
\useinnertheme[shadow=true]{rounded}
%\setbeamertemplate{footline}{\insertframenumber/\inserttotalframenumber}
%\useoutertheme{infolines}
%\setbeamertemplate{headline}{} % removes the headline that infolines inserts

%\usetheme{boxes}
%\usepackage{amsmass}
%\usepackage{amssymb,amsfonts,url}

\usepackage{algorithm}
\usepackage{algorithmic}

\usepackage{graphicx}
\graphicspath{{Problems/}}

\usepackage{tikz}
\usepackage{verbatim}
\usepackage{pgfplots}
\usepackage{verbatim}
\usetikzlibrary{arrows,shapes}
\usepackage{color}

\usepackage{multirow}
\usetheme{Frankfurt}
\usecolortheme{default}
\setbeamertemplate{frametitle}[default][left]

\usepackage{graphicx}
%\graphicspath{{./fig/}}
\usepackage{float}
%\usepackage{color}

\usepackage{cases}
\usepackage{mathrsfs}

\usepackage{url}

\newcommand{\rmnum}[1]{\romannumeral#1}
\newcommand{\Rmnum}[1]{\uppercase\expandafter{\romannumeral#1}}



\begin{document}
\begin{CJK}{UTF8}{gkai}
	
\title{工作汇报}
\author{寿林钧} 
\institute{中科院计算所} 
\date{\today} 

	\begin{frame}
		\titlepage
	\end{frame}



\begin{frame}
	\frametitle{Strain Smoothing: Connectivity}	
	\begin{itemize}
	\item 循环每一个几何单元,建立新的 connectivity(与几何单元相交的所有数学网格的节点).
	\item 在每一个高斯点处计算$\frac{\partial \tilde{N_I}}{\partial x_i}$。
\end{itemize}		

	\begin{figure}
		\centering
		\includegraphics[width=1.2in]{./ljshou_figures/quad_4.eps}
		\caption{New connectivity in Quad4}
	\end{figure}
\end{frame}

\begin{frame}
	\frametitle{Strain Smoothing: Corrected nodal derivatives}
	\begin{align*}
		&\mathbf{q(x)}=[1 \quad x \quad y \quad z]^{\text{T}} \\
		\\
		&\int_{\Omega_S}\mathbf{q(x)}\left(\frac{\partial \tilde{N_I}}{\partial x_i} - \frac{\partial N_I}{\partial x_i}\right)d\Omega = \mathbf{0} \\
		&\int_{\Omega_S}\mathbf{q(x)}\frac{\partial \tilde{N_I}}{\partial x_i}d\Omega = 
		\int_{\Omega_S}\mathbf{q(x)}\frac{\partial {N_I}}{\partial x_i}d\Omega \\
		&\int_{\Omega_S}\mathbf{q(x)}\frac{\partial \tilde{N_I}}{\partial x_i}d\Omega = 
		\int_{\Gamma_S}N_I(\mathbf{x})\mathbf{q(x)}n_i d\Gamma - 
		\int_{\Omega_S}N_I(\mathbf{x})\mathbf{q}_{,i}(\mathbf{x})d\Omega
	\end{align*}
\end{frame}
	
\begin{frame}
	\frametitle{Strain Smoothing: Integration scheme}
	\begin{itemize}
		\item 四面体内部布4个高斯积分点
		\item 四面体面上布7个高斯积分点
	\end{itemize}
	\begin{columns}
		\begin{column}[pos]{0.5\textwidth}
			\begin{figure}
				\includegraphics[width=2.0in]{./ljshou_figures/strain_smoothing_qpoints.eps}
			\end{figure}
		\end{column}
			\begin{column}[pos]{0.5\textwidth}
			\tiny
			\begin{align*}
				&\mathbf{q_{,x}(x)}=[0 \quad 1 \quad 0 \quad 0]^{\text{T}} \\
				\\
				&\int_{\Omega_S}\tilde{N}_{I,x}(\mathbf{x})d\Omega = 					\int_{\Gamma_S}N_I(\mathbf{x})n_xd\Gamma \\
				&\int_{\Omega_S}\tilde{N}_{I,x}(\mathbf{x})xd\Omega = \int_{\Gamma_S}N_I(\mathbf{x})xn_xd\Gamma 
										- \int_{\Omega_S}N_I(\mathbf{x})d\Omega \\
				&\int_{\Omega_S}\tilde{N}_{I,x}(\mathbf{x})yd\Omega = \int_{\Gamma_S}N_I(\mathbf{x})yn_xd\Gamma \\
				&\int_{\Omega_S}\tilde{N}_{I,x}(\mathbf{x})zd\Omega = \int_{\Gamma_S}N_I(\mathbf{x})zn_xd\Gamma \\
			\end{align*}
		\end{column}
	
	\end{columns}
	
\end{frame}

\begin{frame}
	\frametitle{效率问题}
	\begin{itemize}
		\item 新的连通性增加了带宽,导致集成刚度矩阵耗时增加。
		\item Worst case: 连通性从8个节点增加到27个节点,带宽增加3.3倍。
	\end{itemize}
\end{frame}

\begin{frame}
	\frametitle{Strain Smoothing}			
	\begin{enumerate}
		\item 体高斯积分点落在数学单元面上	
		\item 面高斯积分点落在数学单元面上
	\end{enumerate}

	\begin{figure}
		\centering
		\includegraphics[width=1.5in]{./ljshou_figures/quad_4_corner_case.eps}
		\caption{Corner Cases: qpoints located on facets of cube}
	\end{figure}

\end{frame}


%\section{Thanks}
\begin{frame}	
	%\Huge{\centerline{Thank You!}}
	\begin{block}
		\Huge{\centerline{Thank you}}
		\centerline{shoulinjun@ncic.ac.cn}
	\end{block}
\end{frame}
		
\end{CJK}
\end{document}
