%\documentclass[slidestop,compress,mathserif,blue]{beamer}


\documentclass[mathserif]{beamer}
\usepackage[noindent,nofonts]{ctexcap} 
%\setCJKmainfont[BoldFont=STHeiti]{STXihei} 
\setCJKmainfont[BoldFont=STZhongsong]{STZhongsong} 
\usepackage{beamerthemeshadow}
\usepackage{beamerthemesplit}
%\usetheme{shadow}
\usecolortheme{default}
\setbeamertemplate{footline}[frame number]
\useinnertheme[shadow=true]{rounded}
%\setbeamertemplate{footline}{\insertframenumber/\inserttotalframenumber}
%\useoutertheme{infolines}
%\setbeamertemplate{headline}{} % removes the headline that infolines inserts

%\usetheme{boxes}
%\usepackage{amsmass}
%\usepackage{amssymb,amsfonts,url}

\usepackage{algorithm}
\usepackage{algorithmic}

\usepackage{graphicx}
\graphicspath{{Problems/}}

\usepackage{tikz}
\usepackage{verbatim}
\usepackage{pgfplots}
\usepackage{verbatim}
\usetikzlibrary{arrows,shapes}

\usepackage{multirow}
\usetheme{Frankfurt}
\usecolortheme{default}
\setbeamertemplate{frametitle}[default][left]

\usepackage{graphicx}
%\graphicspath{{./fig/}}
\usepackage{float}
%\usepackage{color}

\usepackage{cases}
\usepackage{mathrsfs}

\usepackage{url}

\newcommand{\rmnum}[1]{\romannumeral#1}
\newcommand{\Rmnum}[1]{\uppercase\expandafter{\romannumeral#1}}

\title{课题组2:基因组序列拼接进展汇报}
\author{ 卜$\ $东$\ $波 } 
\institute{中科院计算所} 
\date{\today} 


\begin{document}
	

	\begin{frame}
			\titlepage
	\end{frame}
	
	\begin{frame}
			\frametitle{基因组序列拼接简介}
                          \begin{figure}
				\centering
				\includegraphics[width=6in]{Genome.eps}
			\end{figure}
	\end{frame}

	\begin{frame}
			\frametitle{基因组序列拼接简介}
                          \begin{figure}
				\centering
				\includegraphics[width=6in]{ShotgunAssembly.eps}
			\end{figure}
	\end{frame}

	\begin{frame}
			\frametitle{基因组序列拼接:一个例子}
                          \begin{figure}
				\centering
				\includegraphics[width=4in]{AssemblyExample.eps}
			\end{figure}

	\end{frame}
	
	\begin{frame}
			\frametitle{基因组序列拼接:一个比喻}
                          \begin{figure}
				\centering
				\includegraphics[width=2in]{GenomeBookStrips.eps}
			\end{figure}
	\end{frame}

	\begin{frame}
		\frametitle{经典模型:de Bruijn图$\Rightarrow$欧拉回路}
                          \begin{figure}
				\centering
				\includegraphics[width=4.5in]{EulerianPath.eps}
			\end{figure}
		
	\end{frame}

	\begin{frame}
		\frametitle{困难:重复序列(Repeat)}
                          \begin{figure}
				\centering
				\includegraphics[width=4.5in]{./figures/RepeatsInEulerPath.eps}
			\end{figure}
		
	\end{frame}

	\begin{frame}
		\frametitle{测序仪读长比较}
                          \begin{figure}
				\centering
				\includegraphics[width=4.5in]{GenomeSequencer.eps}
			\end{figure}
		
	\end{frame}

	
	
	
\begin{frame}
		\frametitle{Schematic overview of the assembly algorithm of SOAPdenovo}
			\begin{figure}
				\centering
				\includegraphics[width=2in]{./figures/soap.eps}
				\caption{\tiny Li et al, De novo assembly of human genomes with massively parallel short read sequencing, Genome Research, Dec. 17, 2010}
			\end{figure}
	\end{frame}
	
	\begin{frame}
		\frametitle{SOAPdenovo's methods for genome assembly}
		\begin{itemize}
			\item Preprocessing the sequencing data
			\item Contiging
			\begin{itemize}
				\item Solving tiny repeats
				\begin{figure}
					\centering
					\includegraphics[width=2in]{./figures/tiniyrepeat.eps}
					%\caption{\tiny Li et al, De novo assembly of human genomes with massively parallel short read sequencing, Genome Research, Dec. 17, 2010}
				\end{figure}
				\item Removing tips
				\begin{figure}
					\centering
					\includegraphics[width=2in]{./figures/tipremove.eps}
					%\caption{\tiny Li et al, De novo assembly of human genomes with massively parallel short read sequencing, Genome Research, Dec. 17, 2010}
				\end{figure}
				\item Merging bubbles
				\begin{figure}
					\centering
					\includegraphics[width=2in]{./figures/bubbleremove.eps}
					%\caption{\tiny Li et al, De novo assembly of human genomes with massively parallel short read sequencing, Genome Research, Dec. 17, 2010}
				\end{figure}
			\end{itemize}
		\end{itemize}
	\end{frame}
	\begin{frame}
		\frametitle{SOAPdenovo's method for genome assembly}
		\begin{itemize}
			\item Scaffolding
			\begin{itemize}
				\item Extract unambiguously linear paths from the graph to construct scaffolds
				\item Extend the scaffolds starting with short paired-ends and then iterate the scaffolding process, step by step, using longer insert size paired-ends
			\end{itemize}
			\begin{figure}
				\centering
				\includegraphics[width=2in]{./figures/scaffold.eps}
				%\caption{\tiny Li et al, De novo assembly of human genomes with massively parallel short read sequencing, Genome Research, Dec. 17, 2010}
			\end{figure}
			\item Gap closure
			\begin{figure}
				\centering
				\includegraphics[width=2in]{./figures/gapfilling.eps}
				%\caption{\tiny Li et al, De novo assembly of human genomes with massively parallel short read sequencing, Genome Research, Dec. 17, 2010}
			\end{figure}
		\end{itemize}
	\end{frame}
	

	
		\begin{frame}
			\frametitle{Limitation of traditional method}
			\begin{itemize}
				\item Traditional method merge the repeats first and then try to split them.
				\item Inexact repeats make the de Bruijn graph entangled. It is not easy to untangle the area by traditional method.
			\end{itemize}
		\end{frame}
		\begin{frame}
			\frametitle{de Bruijn graph}
			\begin{figure}
				\centering
				\includegraphics[width=3.5in]{./figures/Diagram2.eps}
			\end{figure}
		\end{frame}
		\begin{frame}
		\begin{figure}
				\centering
				\includegraphics[width=3.5in]{./figures/Eular.png}
				\caption{\tiny Pevzner et al, An Eulerian path approach to DNA fragment assembly, PNAS, August 14, 2001} 
			\end{figure}
		\end{frame}
	
	%%\section{Our motivation}
		\begin{frame}
			\frametitle{Motivation}
			\begin{itemize}
				\item Most of the popular genome assembly algorithms employ a ``bottom-up" strategy, that is, reads are first merged into contigs via tip removing and bubbles merging in a de Bruijn graph, and the contigs are further merged into scaffolds with the helps of pair-ends information.  
				\item However, the “bottom-up’’ strategy usually suffers from the entangled repeats. How to extract sequence from the entangled repeats remains as a challenge.
				\item We proposed a “top-down” strategy, i.e. the framework of the whole genome is generated first by using paired k-mers information of the unique regions, and the gaps are filled by using bounded BFS. 
			\end{itemize}
		\end{frame}

	
	%%\section{A top-down Method}
		\begin{frame}
			\frametitle{A Top-down Method: ARTS}
			\begin{itemize}
				\item Convert de Bruijn Graph to condensed de Bruijn Graph.
				\item Estimate the copy number of every condensed edge.
				\item Map paired-reads to the condensed edges and score the links, then remove the low score links.
				\item Remove false negative repeats and linearize the condensed edges.
				\item Gap closuring by distance-bounded BFS. 
			\end{itemize}
		\end{frame}
		
		\begin{frame}
		\frametitle{A Top-down Method: ARTS}
			\begin{figure}
				\centering
				\includegraphics[width=3.5in]{./figures/ARTS.png}
			\end{figure}
			
		\end{frame}				
		
		\begin{frame}
	\frametitle{Copy Number Estimation using Minimum Cost Flow Method}
	
	ARTS adopt a maximum likelihood framework to estimate the copy number of sequence.\\
	In this setting, ARTS follow a directed network flow-based algorithm, which takes advantage of high coverage to estimate the copy counts of repeats in a genome accurately.
	The accuracy of estimation on simulated E.coli reads is 99.99\% and that of real reads is 97\%.

	\end{frame}
	
		\begin{frame}
		\frametitle{Scoring Links between Contigs}
		ARTS adopt a generative model; Here, ARTS suppose the distribution of the count of paired kmers between two condensed edge follows a Poisson $P_{\lambda}$, then the log likelihood evaluates the quality of the link between two condensed edges, in which the sore of the links caused by chimeric reads is very low. 
		\begin{eqnarray}
   							\lambda = \int _{d1} \int _{d2} KG_{\mu , \sigma}(y+d-x)dxdy
    		\end{eqnarray}\\
    		The insert size of paired reads follow a Gaussian distribution.\\
    		$G_{\mu , \sigma}$ is the probability of a Gaussian distribution with parameter $\mu , \sigma$ .\\
    		$d_{1}$ is the length of left edge and $d_{2}$ is the length of right edge.\\
    		$K$ is the mean pair-kmer number of every site. 
    		
		\begin{eqnarray}
   							P(c|d) = P_{\lambda}(c)
    		\end{eqnarray}\\
    		
    		$P_{\lambda}(c)$ is the probablity of a Poisson distribution with parameter $\lambda$. 
    		
		\end{frame}
		
		\begin{frame}
		\frametitle{Linearizing the Condensed Edges}
			Reconstructing a sparse signal x approximately from noisy data
			\begin{eqnarray}
			b = Ax + z, 
			\end{eqnarray}
			
			Assuming that z has norm less than some error tolerance $\epsilon$.
			
			Formalization:
		\[
		\begin{array}{rrrrrrrrlr}
		 \min & \sum _{i}  & |{\epsilon}_{i}| &  &\\
		 s.t. &  AX & + & \epsilon & = & b\\
		\end{array} \nonumber
		\]
		
		A is the arc matrix and b is the distance between two condensed edges.
		\end{frame}	
	
		\begin{frame}
    			\frametitle{Distance Estimation}
    				
    				The probability of Mapped paired-kmer\\
    						\begin{eqnarray}
   							 P(d_{1i},d_{2i}|d)=G_{\mu ,\sigma}(d_{1i} + d_{2i} + d + 2r)\frac{1}{L}
    						\end{eqnarray}\\
    				Likelihood:    				
    				
    						\begin{eqnarray}
   							 L(pair reads|d)=\prod_{i=1}^{n}P(d_{1i},d_{2i}|d)\times(1-\sum _{s}G_{\mu , \sigma}(s)\frac{w_d(s)}{L})^{R-n}
    						\end{eqnarray}\\
    				Objective:\\
    						\begin{eqnarray}
    							\bar{d}_{ij} = argmax _d \quad L(data|d)
    						\end{eqnarray}\\
    				
    				
    		\end{frame}
		
	%%\section{Result}
		\begin{frame}
			\frametitle{Result}
			ARTS shows better performance than SOAP on several genomes, especially for bacterial genomes containing small repeats.
			\begin{figure}
				\centering
				\includegraphics[width=4in]{./11.png}
			\end{figure}
		\end{frame}

	%\section{Future Work}
		\begin{frame} 
			\frametitle{Future Work}
			\begin{itemize}
				\item Improve the accuracy of copy number estimation. 	
				\item Reduce memory and time usage of ARTS.
				\item Metagenome assembly
				\item Combining 2nd and 3rd generation sequencing data 
				%\item Solve the circles caused by repeats
			\end{itemize}	
		\end{frame}
	%\section{Acknowledgement}
		\begin{frame}
			\frametitle{Acknowledgement}
			\begin{itemize}
				\item Xingwu Liu, ICT, CAS
				\item Shiwei Sun, ICT, CAS
				\item Shuai Cheng Li, CityU
				\item Cheng Wang, AMSS, CAS
				\item Bin Ling, ICT, CAS
				\item Chunlin Huang, ICT, CAS
				\item Chao Wang, ICT, CAS
			\end{itemize}
		\end{frame}
	%\section{Thanks}
		\begin{frame}	
			\Huge{\centerline{Thank You!}}
		\end{frame}
		
\end{document}
