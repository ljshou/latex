%\documentclass[slidestop,compress,mathserif,blue]{beamer}
\documentclass[mathserif]{beamer}
%\usepackage[noindent,nofonts]{ctexcap} 
\usepackage{CJK} 
%\setCJKmainfont[BoldFont=STHeiti]{STXihei} 
%\setCJKmainfont[BoldFont=STZhongsong]{STZhongsong} 
\usepackage{beamerthemeshadow}
\usepackage{beamerthemesplit}
%\usetheme{shadow}
\usecolortheme{default}
\setbeamertemplate{footline}[frame number]
\useinnertheme[shadow=true]{rounded}
%\setbeamertemplate{footline}{\insertframenumber/\inserttotalframenumber}
%\useoutertheme{infolines}
%\setbeamertemplate{headline}{} % removes the headline that infolines inserts

%\usetheme{boxes}
%\usepackage{amsmass}
%\usepackage{amssymb,amsfonts,url}

\usepackage{algorithm}
\usepackage{algorithmic}

\usepackage{graphicx}
\graphicspath{{Problems/}}

\usepackage{tikz}
\usepackage{verbatim}
\usepackage{pgfplots}
\usepackage{verbatim}
\usetikzlibrary{arrows,shapes}
\usepackage{color}

\usepackage{multirow}
\usetheme{Frankfurt}
\usecolortheme{default}
\setbeamertemplate{frametitle}[default][left]

\usepackage{graphicx}
%\graphicspath{{./fig/}}
\usepackage{float}
%\usepackage{color}

\usepackage{cases}
\usepackage{mathrsfs}

\usepackage{url}

\newcommand{\rmnum}[1]{\romannumeral#1}
\newcommand{\Rmnum}[1]{\uppercase\expandafter{\romannumeral#1}}



\begin{document}
\begin{CJK}{UTF8}{gkai}
	
\title{程序实现}
\author{寿林钧} 
\institute{中科院计算所} 
\date{\today} 

\begin{frame}
	\titlepage
\end{frame}

%%%%%%%%%%%%%%%%%%%%%%%%%%%%%%%%%%%%%%%%%%%%%%%%%%%%%%%%%%%
%%%%%%%%%%%%%%%%%%%%%%%% 背景网格实现 %%%%%%%%%%%%%%%%%%%%%%%
\begin{frame}
	\frametitle{实现:背景网格粘接}
	组合网格具有不协调性,如何进行网格粘接?
		\begin{itemize}
			\item 利用背景网格构造插值函数
		\end{itemize}
		\begin{figure}
			\centering
			\includegraphics[width=4in]{./ljshou_figures/math_glue_gauss.eps}
			%\caption{}
		\end{figure}
\end{frame}

\begin{frame}
	\frametitle{实现:背景网格粘接}
	弱形式积分策略
	\begin{itemize}
		\item 直接积分法
		\item 数学网格切割法
		\begin{itemize}
			\item 精确
		\end{itemize}
		\item Strain smoothing
	\end{itemize}
\end{frame}

\begin{frame}
	\frametitle{实现:背景网格粘接}
	四面体和六面体切割,再剖分
	\begin{enumerate}
		\item 求所交的凸多面体的顶点
		\item Delaunay Triangulation
		图. Delaunay triangulation 示意图
	\end{enumerate}
	图.切割速度对比表格
	图.实际切割图,仍然可以很好地描述边界
\end{frame}

\begin{frame}
	\frametitle{实现:背景网格粘接}
	利用相交性测试,处理特例
	\begin{itemize}
		\item tet in cube
		\item cube in tet
		\item normal intersection
		\item no intersection
	\end{itemize}
	图(四面体六面体内部,外部)
\end{frame}

\begin{frame}
	\frametitle{实现:背景网格粘接}
	仅仅需要对边界数学单元进行切割
	\begin{itemize}
		\item 边界数学单元定义:不是完全落于材料内部的单元
		\item 内部数学单元直接布高斯积分点即可
		\item 极大地减少计算量
	\end{itemize}				
	\begin{figure}
			\centering
			\includegraphics[width=3in]{./ljshou_figures/mesh_cut.eps}
			%\caption{}
	\end{figure}
	\begin{block}{}
		\bf \color{blue} 如何快速找到边界数学单元?
	\end{block}
\end{frame}

\begin{frame}
	\frametitle{实现:背景网格粘接}
	如何快速找到边界数学单元
	\begin{itemize}
		\item 利用网格切割,完全落于材料内部的单元(精确,但较为耗时)
		\item 对整个计算区域建模(不适用于大规模计算)
		\item 与边界几何单元相交的数学单元
		\begin{enumerate}
			\item 找边界几何单元
			\item 利用相交性测试求与边界几何单元相交的数学单元
		\end{enumerate}
	\end{itemize}

\end{frame}

\begin{frame}
	\frametitle{实现:背景网格粘接}
	如何快速找到边界几何单元
	\begin{itemize}
		\item 找 boundary edge(2D) 或 boundary face(3D)
		\item boundary edge/face 仅仅由一个几何单元享有
	\end{itemize}
	\begin{figure}
		\includegraphics[width=3in]{ljshou_figures/boundary.eps}
		\caption{几何网格的边界边定义(2D)}
	\end{figure}
	\begin{block}{}
		\bf \color{blue} 邻居为空的单元为边界几何单元
	\end{block}
\end{frame}

\begin{frame}
	\frametitle{实现:背景网格粘接}
	求边界几何单元算法
	图,例子
	\begin{columns}
		\begin{column}[pos]{0.5\textwidth}
		\begin{figure}
			\includegraphics[width=2in]{ljshou_figures/cube_boundary_node.eps}
			%\caption{}
		\end{figure}
		\end{column}
		
		\begin{column}[pos]{0.5\textwidth}
		\begin{figure}
			\includegraphics[width=2in]{ljshou_figures/cube_boundary_node.eps}
			%\caption{}
		\end{figure}
		\end{column}
	\end{columns}
\end{frame}

%\section{Thanks}
\begin{frame}	
	\begin{block}
		\Huge{\centerline{Thank you}}
		\centerline{shoulinjun@ncic.ac.cn}
	\end{block}
\end{frame}
		
\end{CJK}
\end{document}
