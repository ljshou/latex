%\documentclass[slidestop,compress,mathserif,blue]{beamer}
\documentclass[mathserif]{beamer}
%\usepackage[noindent,nofonts]{ctexcap} 
\usepackage{CJK} 
%\setCJKmainfont[BoldFont=STHeiti]{STXihei} 
%\setCJKmainfont[BoldFont=STZhongsong]{STZhongsong} 
\usepackage{beamerthemeshadow}
\usepackage{beamerthemesplit}
%\usetheme{shadow}
\usecolortheme{default}
\setbeamertemplate{footline}[frame number]
\useinnertheme[shadow=true]{rounded}
%\setbeamertemplate{footline}{\insertframenumber/\inserttotalframenumber}
%\useoutertheme{infolines}
%\setbeamertemplate{headline}{} % removes the headline that infolines inserts

%\usetheme{boxes}
%\usepackage{amsmass}
%\usepackage{amssymb,amsfonts,url}

\usepackage{algorithm}
\usepackage{algorithmic}

\usepackage{graphicx}
\graphicspath{{Problems/}}

\usepackage{tikz}
\usepackage{verbatim}
\usepackage{pgfplots}
\usepackage{verbatim}
\usetikzlibrary{arrows,shapes}
\usepackage{color}

\usepackage{multirow}
\usetheme{Frankfurt}
\usecolortheme{default}
\setbeamertemplate{frametitle}[default][left]

\usepackage{graphicx}
%\graphicspath{{./fig/}}
\usepackage{float}
%\usepackage{color}

\usepackage{cases}
\usepackage{mathrsfs}

\usepackage{url}

\newcommand{\rmnum}[1]{\romannumeral#1}
\newcommand{\Rmnum}[1]{\uppercase\expandafter{\romannumeral#1}}



\begin{document}
\begin{CJK}{UTF8}{gkai}
	
\title{面向千万亿次计算的高可扩展大型非结构化网格有限元算法研究}
\author{导师:田荣 \\ 学生:寿林钧} 
\institute{中科院计算所高性能中心} 
\date{\today} 

	\begin{frame}
		\titlepage
	\end{frame}

    \begin{frame}
        \frametitle{提纲}
        \begin{itemize}
            \item 研究背景
            \item 国内外研究现状分析
            \item 研究目标、研究内容和拟解决的关键技术 
		  \begin{itemize}
		  	\item 组合网格有限元
			\item 大应变计算
		  \end{itemize}
            \item 拟采取的研究方法、技术路线和可行性分析
            \item 已有工作基础
	    \item 研究计划和预期进展
            \item 参考文献
        \end{itemize}
	\end{frame}
	
	    \begin{frame}
        \frametitle{提纲}
        \begin{itemize}
            \color{gray} 
            \item {\color{blue}研究背景}
            \item 国内外研究现状分析
            \item 研究目标、研究内容和拟解决的关键技术 
		  \begin{itemize}
		    \color{gray}
		  	\item 组合网格有限元
			\item 大应变计算
		  \end{itemize}
            \item 拟采取的研究方法、技术路线和可行性分析
            \item 已有工作基础
	    \item 研究计划和预期进展
            \item 参考文献
        \end{itemize}
	\end{frame}
	
	\begin{frame}
	  \frametitle{研究背景: “千万亿次计算”}
	  \begin{itemize}
		\item “千万亿次计算”现状
		   \begin{itemize}
			\item Top500 最新的统计显示,理论峰值1Petaflops以上的超级计算机已经达到55台
		   \end{itemize} 
		   
		\item “千万亿次计算”的关键问题
		   \begin{itemize}
			\item 程序的可扩展能力
			\item 程序的可容错能力
			\item 浮点效率
			\item 异构协同
		   \end{itemize}
	  \end{itemize}
	\end{frame}

	\begin{frame}
		\frametitle{研究背景:主要技术挑战(1)}
		{\color{red}\bf 大型复杂非结构化网格有限元模型的建立}
		\begin{itemize}
		  \item 大规模有限元的成功应用仍以结构化网格为主,非结构化网格应用严重受到网格生成困难的制约
		  \item IO数据量的挑战
		  \begin{itemize}
			\item 100亿自由度网格,输入数据 ~678GB(二进制格式)
		  \end{itemize}
		\end{itemize}
			\begin{figure}
				\centering
				\includegraphics[width=1.5in]{./ljshou_figures/nuclear_plant1.eps}
				\includegraphics[width=1.5in]{./ljshou_figures/nuclear_plant2.eps}
			\end{figure}
	\end{frame}

	\begin{frame}
		\frametitle{研究背景:主要技术挑战(2)}
		\begin{columns}
			\begin{column}[pos]{0.4\textwidth}
				\begin{figure}
					\includegraphics[width=1.2in]{./ljshou_figures/traditional_fem.eps}
				\end{figure}
			\end{column}
			
			\begin{column}[pos]{0.6\textwidth}
				\begin{itemize}
		  			\item 大多历史遗产代码(特别是国外商业程序)大多诞生于PC时代,算法架构设计没有、也无法考虑目前的海量并行需求
		  		\begin{itemize}
					\item 面对未来海量并行,架构性矛盾突出,这是目前商业程序不可扩展的根本原因
		  		\end{itemize}
		  		\item 面向P级E级计算,应用数学算法的设计应该考虑并行计算本身的特点,如对并行性,数据移动复杂性,甚至容错性的重视
		\end{itemize}
			\end{column}
		\end{columns}
		\begin{block}{}
			\color{red}\bf 历史遗产并行架构在前、后处理上的串行瓶颈
		\end{block}
	\end{frame}

	    \begin{frame}
        \frametitle{提纲}
        \begin{itemize}
            \color{gray} 
            \item 研究背景
            \item {\color{blue}国内外研究现状分析}
            \item 研究目标、研究内容和拟解决的关键技术 
		  	\begin{itemize}
		    		\color{gray}
		  		\item 组合网格有限元
				\item 大应变计算
		  	\end{itemize}
            \item 拟采取的研究方法、技术路线和可行性分析
            \item 已有工作基础
	    \item 研究计划和预期进展
            \item 参考文献
        \end{itemize}
	\end{frame}
	
		\begin{frame}
		\frametitle{相关研究:协调网格生成}
		\begin{itemize}
		  \small
		  \item Conforming assembly meshing (CUBIT (Blacker,1994); Assembly meshing (Tautges,2001); Tolerant imprinting (Clark et al., 2008); Mesh matching (Staten et al.,2008)(协调的组合网格生成)
			\begin{itemize}
			\scriptsize
			\item Mortar element Methods (Bernardi, Maday and Patera(1990); Belgacem and Maday(1997))
			\item FETI methods (Farhat and Roux(1991); Farhat and Geradin(1992)) 
			\item MPCs, tied contact, and gap elements in commercial software
		  	\end{itemize}
		  	\begin{figure}
		  		\includegraphics[width=2in]{./ljshou_figures/mesh_matching.png}
		  	\end{figure}
		\end{itemize}
	\end{frame}

	\begin{frame}
		\frametitle{相关研究:组合网格粘接}
		\begin{itemize}
		  %\small
		  \item Non-conforming assembly mesh gluing using Lagrange multipliers(非协调组合网格的拉格朗日乘子粘接)
		  \begin{itemize}
		  	\scriptsize
			\item Mortar element Methods (Bernardi, Maday and Patera(1990); Belgacem and Maday(1997))
			\item FETI methods (Farhat and Roux(1991); Farhat and Geradin(1992)) 
			\item MPCs, tied contact, and gap elements in commercial software
		  \end{itemize}
		  \item 缺点: 刚度矩阵的非正定性,造成求解的困难
		\end{itemize}
	\end{frame}

	
	
	\begin{frame}
        \frametitle{提纲}
        \begin{itemize}
            \color{gray} 
            \item 研究背景
            \item 国内外研究现状分析
            \item {\color{blue}研究目标、研究内容和拟解决的关键技术 }
		  	\begin{itemize}
		    		\color{gray}
		  		\item 组合网格有限元
				\item 大应变计算
		  	\end{itemize}
            \item 拟采取的研究方法、技术路线和可行性分析
            \item 已有工作基础
	    \item 研究计划和预期进展
            \item 参考文献
        \end{itemize}
	\end{frame}
	
		\begin{frame}
			\frametitle{研究目标}
			\begin{itemize}
				\item 理论研究:
				\begin{itemize}
					\item 发挥网格扭曲不敏感的优势,实现高精度大应变计算
				\end{itemize}
				\item 程序开发:
				\begin{itemize}
					\item 对比分析三种网格粘结算法
					\item 采用"半"完全拉格朗日方法,开发高精度大应变有限元模拟程序
				\end{itemize}
			\end{itemize}
			
		\end{frame}
		
		
		\begin{frame}
		\frametitle{研究内容《一》:组合网格有限元}
			\begin{itemize}
				\item 组合化网格优点: 
					\begin{itemize}
						\item 网格生成简单
						\item 网格局部加密、重构等操作更加容易施加
						\item 前后处理具有高度并行性
					\end{itemize}
				\item 技术难点:
					\begin{itemize}
						\item 如何将网格“粘结”起来?保证内部区域的连续性:
						\begin{equation}
							{\bf u}^i = {\bf u}^j \text{ on } \Gamma_{ij} = \Omega_i \cap \Omega_j
						\end{equation}
					\end{itemize}
					\begin{figure}
						\centering
						\includegraphics[width=2in]{./ljshou_figures/combined_mesh.eps}
						%\caption{}
					\end{figure}
			\end{itemize}
		\end{frame}	
	
		\begin{frame}
			\frametitle{组合网格FEM与传统FEM对比}
				\begin{figure}
					\centering
					\includegraphics[width=3in]{./ljshou_figures/fem_compare.eps}
					\caption{并行有限元分析:(a)协调网格 (b)组合网格}
			\end{figure}
		\end{frame}
				
		%\begin{frame}
		%	\frametitle{组合网格FEM与传统FEM对比}
		%\end{frame}		
		%\begin{frame}
		%	\frametitle{组合网格FEM与传统FEM对比}
		%\end{frame}		
	
		\begin{frame}
			\frametitle{研究内容《二》:大应变计算}
			\begin{itemize}
				\item 历史相关
				\item 采用“半”完全拉格朗日方法
			\end{itemize}
				\begin{figure}
					\centering
					\includegraphics[width=3in]{./ljshou_figures/step.png}
					\caption{step Total Lagrangian Formulation}
			\end{figure}
		\end{frame}	
	
		\begin{frame}
			\frametitle{变形信息映射}
			\begin{itemize}
				\item 几步更新一次历史数据
				\item 当网格扭曲到一定程度时再更新
			\end{itemize}
				\begin{figure}
					\centering
					\includegraphics[width=3in]{./ljshou_figures/remesh.eps}
			\end{figure}
		\end{frame}	
				
		\begin{frame}
			\frametitle{变形信息映射}
			\begin{itemize}
				\item 变形信息(例如变形梯度 $F^n$)存储在高斯积分点上
				\item 更新时,需要将变形信息从旧的高斯点映射到新的高斯点上
				\item 张量插值问题(技术难点)
			\end{itemize}
	\begin{figure}
		\centering
		\includegraphics[width=3in]{./ljshou_figures/projection.eps}
		%\caption{}
	\end{figure}
		\end{frame}	
			
	\begin{frame}
        \frametitle{提纲}
        \begin{itemize}
            \color{gray} 
            \item 研究背景
            \item 国内外研究现状分析
            \item 研究目标、研究内容和拟解决的关键技术
		  	\begin{itemize}
		    		\color{gray}
		  		\item 组合网格有限元
				\item 大应变计算
		  	\end{itemize}
            \item {\color{blue}拟采取的研究方法、技术路线和可行性分析}
            \item 已有工作基础
	    \item 研究计划和预期进展
            \item 参考文献
        \end{itemize}
	\end{frame}
		
			
		\begin{frame}
			\frametitle{技术难点及拟采用的实验方案}
			\begin{itemize}
				\item 网格粘结:如何保证内部界面的连续性?
				\begin{itemize}
					\item 背景网格粘结
					\item 全局无网格粘结
					\item 界面无网格粘结
				\end{itemize}
			
				\item 超大应变计算如何保证精度?

				\begin{itemize}
					\item “半”完全拉格朗日方法
					\item 背景网格法
				\end{itemize}
						
			\end{itemize}

		\end{frame}	
		
	\begin{frame}
		\frametitle{背景网格粘结:Basic Idea}
		\begin{columns}
			\begin{column}[pos]{0.5\textwidth}
				Single mesh:				
				\begin{itemize}
					\item approximation
					\item geometry/boundary representation
					\item quadrature
				\end{itemize}
			\end{column}
			\begin{column}[pos]{0.6\textwidth}
				Geometric mesh: 
				\begin{itemize}
					\item geometry/boundary representation
					\item quadrature
				\end{itemize}
				Math grid:
				\begin{itemize}
					\item approximation			
				\end{itemize}
			\end{column}
		\end{columns}
				\begin{figure}
					\centering
					\includegraphics[width=2in]{./ljshou_figures/conforming_fem.eps}
					\includegraphics[width=2in]{./ljshou_figures/math_glue.eps}
					%\caption{}
			\end{figure}
			
			\begin{block}{}
				\color{red}\bf 积分域和插值域分开,强调网格的灵活性
			\end{block}
	\end{frame}
		
		\begin{frame}
			\frametitle{背景网格粘结: 基本概念}
				\begin{figure}
					\centering
					\includegraphics[width=3in]{./ljshou_figures/concepts.eps}
					\caption{背景网格法中的基本概念}
			\end{figure}
		\end{frame}
		
		\begin{frame}
			\frametitle{背景网格粘结: 计算过程}
			\begin{itemize}
				\item 产生高斯点后,物理网格并不参与计算
				\item 自由度是数学网格节点的自由度
			\end{itemize}
			因此,内部界面不连续问题很自然地解决了。
				\begin{figure}
					\centering
					\includegraphics[width=4in]{./ljshou_figures/math_glue_gauss.eps}
					%\caption{}
			\end{figure}
		\end{frame}
		
		\begin{frame}
			\frametitle{背景网格粘结:形式化}
			强形式的 PDE 如下,
			\begin{equation}
				\begin{cases}
					\nabla {\bf \sigma} + {\bf f} = 0& \text{in }\Omega\\
					{\bf n}\cdot{\bf \sigma} = {\bf t}& \text{on }\Gamma
				\end{cases}
			\end{equation}
			离散化的材料建立在结构化的背景网格$V$上, $V \supseteq \Omega$
			%\begin{block}{}
				\begin{equation}
					{\bf x} = \sum_{I=1}^{n}N_I{\bf x}_I(t), \quad supp(N) = \{{\bf x} \ni V \supseteq \Omega\}
				\end{equation}
			%\end{block}
		\end{frame}

		\begin{frame}
			\frametitle{背景网格粘结: 弱形式积分策略(技术难点)}
			\begin{itemize}
				\item 直接积分法
				\item 数学网格切割法
				\begin{itemize}
					\item 非常耗时
					\item 高斯点过多,导致刚度矩阵计算不经济
				\end{itemize}
			\end{itemize}
			\begin{figure}
					\centering
					\includegraphics[width=3in]{./ljshou_figures/integration.eps}
					%\caption{}
			\end{figure}
		\end{frame}
		
		\begin{frame}
			\frametitle{背景网格粘结: 弱形式积分策略(技术难点)}
			\begin{itemize}
				\item 剖分仅在材料边界处进行
				\item 内部数学cell直接布高斯积分点
  				\item 极大地减少计算量
			\end{itemize}
			\begin{figure}
					\centering
					\includegraphics[width=3in]{./ljshou_figures/mesh_cut.eps}
					%\caption{}
			\end{figure}
			\begin{block}{}
				\bf \color{blue} 如何快速找到边界数学单元?
			\end{block}
		\end{frame}
		
\begin{frame}
	\frametitle{全局无网格粘结}
	\begin{itemize}
		\item 有三种比较成熟的无网格插值方法:
		\begin{itemize}
			\item 径向基函数(radial basis function)
			\item 径向点插值法(radial point interpolation)
			\item Kringing插值法。
		\end{itemize}
		\item 采用径向基函数方法进行网格粘结
		\begin{eqnarray}
			{\bf u}^{GE}({\bf x})=\sum_{I=1}^{N^{GE}}{\bf \phi}_I({\bf x}){\bf u}_I, \quad {\bf x} \in \Omega^{GE} \\
			\text{其中,} \quad {\bf \phi}_I({\bf x}) = \sum_{i=1}^NR_i({\bf x}){\bf A}_{iI} + \sum_{j=1}^mp_j({\bf x}){\bf B}_{jI} 
			\end{eqnarray}
	\end{itemize}
\end{frame}
		
		\begin{frame}
			\frametitle{全局无网格粘结}
			\begin{itemize}
				\item 所有单元采用RBF插值
				\begin{itemize}
					\item 用积分点影响域内的节点建立连通性,构造插值函数
				\end{itemize}
			\end{itemize}
				\begin{figure}
					\centering
					\includegraphics[width=2in]{./ljshou_figures/global_glue.eps}
					\caption{全局无网格插值}
			\end{figure}
		\end{frame}	

\begin{frame}
	\frametitle{算法流程}
	\begin{enumerate}
		\item 循环所有的几何单元  
		\item 循环单元的高斯积分点
		\item 用影响域内部的节点建立连通性
		\item 在高斯积分点处构建 RBF 插值
		\item 计算单元刚度矩阵并组装
		\item Go to 2
		\item Go to 1
		\item End
	\end{enumerate}
\end{frame}

\begin{frame}
	\frametitle{界面无网格粘结}	
	\small 全局无网格粘结邻居点搜索 $O(N^2)$
	\begin{itemize}
		\item 界面单元(GE)
		\item 有限元单元(FE)
	\end{itemize}
	\begin{figure}
		\centering
		\includegraphics[width=2.0in]{./ljshou_figures/interface_glue1.eps}
		%\caption{}
	\end{figure}
\end{frame}

\begin{frame}
	\frametitle{界面无网格粘结}	
	\begin{itemize}
		\small
		\item 界面单元(GE): 无网格插值
		\begin{equation}
			\small
			{\bf u}^{GE} = \sum_{I=1}^{N^{GE}}{\bf \phi}_I({\bf x}){\bf u}_I, \quad {\bf x} \in \Omega^{GE}
		\end{equation}
		\item 有限元单元(FE): 有限元插值
		\begin{equation}
			\small
			{\bf u}^{FE} = \sum_{I=1}^{N^{FE}}{\bf N}_I({\bf x}){\bf u}_I, \quad {\bf x} \in \Omega^{FE}
		\end{equation}
	\end{itemize}
			\begin{figure}
					\centering
					\includegraphics[width=2.0in]{./ljshou_figures/interface_glue2.eps}
					%\caption{}
			\end{figure}
\end{frame}

\begin{frame}
	\frametitle{大应变计算:变形信息映射}	
	\begin{itemize}
		\item 张量插值问题(技术难点)
		\begin{itemize}
			\item 通过数学网格节点插值(component-wise)
			\item 通过张量的特征向量插值
			\item MLS 进行两套点的插值
			\item Strain Smoothing(避免映射)
		\end{itemize}
	\end{itemize}
	\begin{figure}
		\centering
		\includegraphics[width=3.5in]{./ljshou_figures/projections_math_node.eps}
		%\caption{}
	\end{figure}
\end{frame}

\begin{frame}
	\frametitle{课题创新点}
	\begin{itemize}
		\item 克服网格生成,后处理上的可扩展性瓶颈,实现面向千万亿次超级计算机的高可扩展计算
		\item 新方法将积分域和插值域分离,克服大应变计算时网格扭曲导致精度下降的问题
	\end{itemize}
\end{frame}


	\begin{frame}
        \frametitle{提纲}
        \begin{itemize}
            \color{gray} 
            \item 研究背景
            \item 国内外研究现状分析
            \item 研究目标、研究内容和拟解决的关键技术
		  	\begin{itemize}
		    		\color{gray}
		  		\item 组合网格有限元
				\item 大应变计算
		  	\end{itemize}
            \item 拟采取的研究方法、技术路线和可行性分析
            \item {\color{blue}已有工作基础}
	    \item 研究计划和预期进展
            \item 参考文献
        \end{itemize}
	\end{frame}
	
	
\begin{frame}
	\frametitle{背景网格法的理论分析}
	\begin{itemize}
		\item 泊松方程求解:
		%\begin{equation}
			$-\Delta u = f, \quad in [-1, 1]^3$
		%\end{equation}
		\item 精确解是
		%\begin{equation}
			$u = (1-x^2)(1-y^2)(1-z^2)$
		%\end{equation}
	\end{itemize}
		\begin{figure}
		\centering
		\includegraphics[width=3in]{./ljshou_figures/poisson_test.eps}
		\caption{背景网格法收敛性测试}
	\end{figure}
\end{frame}

\begin{frame}
	\frametitle{背景网格法的理论分析}
	\begin{itemize}
		\item 在组合几何网格上的数值积分
		\begin{itemize}
			\item 直接积分(图1中pGFEM\_1\_1和pGFEM\_1\_2);
			\item 数学网格切割(图1中pGFEM\_out)。
		\end{itemize}
		\item 目前的工作实践证明了方法b是精确、实际可行的方法,确保了方法的理论严密性。
	\end{itemize}
		\begin{figure}
				\centering
				\includegraphics[width=3in]{./ljshou_figures/math_test.eps}
		\end{figure}
\end{frame}

\begin{frame}
	\frametitle{网格扭曲测试---负雅克比网格}
	\begin{figure}
		\centering
		\includegraphics[width=3in]{./ljshou_figures/jacobian_test.eps}
		\caption{大变形分析中典型的广义有限元网格构型(左).}
		\caption{数学网格法,即使在几何网格在严重扭曲的情况下,仍然可以给出与最规则网格有限元相一致的结果(右). }
	\end{figure}
\end{frame}

	\begin{frame}
        \frametitle{提纲}
        \begin{itemize}
            \color{gray} 
            \item 研究背景
            \item 国内外研究现状分析
            \item 研究目标、研究内容和拟解决的关键技术
		  	\begin{itemize}
		    		\color{gray}
		  		\item 组合网格有限元
				\item 大应变计算
		  	\end{itemize}
            \item 拟采取的研究方法、技术路线和可行性分析
            \item 已有工作基础
	    \item  {\color{blue}研究计划和预期进展}
            \item 参考文献
        \end{itemize}
	\end{frame}

\begin{frame}
	\frametitle{预期工作}
	\begin{itemize}
		\item 2014.11- 2015.01 解决弱形式积分问题,变形历史信息映射问题
		\item 2015.01-2015.02 对比分析三种组合网格粘结算法
		\item 2015.02-2015.05 论文写作,以及代码调优
	\end{itemize}
\end{frame}

\begin{frame}
	\frametitle{参考文献}
	\begin{enumerate}
		\tiny
		\item Tian R*, Wu ZD, Wang CW, Scalable FEA on non-conforming assembly mesh. Computer Methods in Applied Mechanics \& Engineering. 2013; 266: 98-111
		\item Tian R*, Meshfree/GFEM in hardware-efficiency prospective. Interaction and multiscale mechanics. DOI:10.12989/imm. 2013.6.2.000, 2013 
		\item Tian R*, Extra-dof-free and linearly independent enrichments in GFEM. Computer Methods in Applied Mechanics \& Engineering. 2013; 266: 1-22
		\item Y. Sudhakar, Wolfgang A. Wall, Quadrature schemes for arbitrary convex/concave volumes and integration of weak form in enriched partition of unity methods. Comput. Methods Appl. Mech. Engrg. 258 (2013) 39–54
		\item Tian R*, Simulation at Extreme-Scale: Co-Design Thinking and Practices. Archive of Computational Methods in Engineering, DOI 10.1007/s11831-014-9095-y, 2014
		\item Rong Tian, Albert C. To, Wing Kam Liu. Conforming local meshfree method. International Journal for Numerical Methods in Engineering 2011; 86(3): 335-357.
		\item Luis Quiroz, Pierre Beckers. Non-conforming mesh gluing in the finite elements method. International Journal for Numerical Methods in Engineering 1995; 38: 2165 – 2184.
		\item Rong Tian, G. Yagawa. Non-matching mesh gluing by meshless interpolation – an alternative to Lagrange multipliers. International Journal for Numerical Methods in Engineering 2007; 71: 473-503.
		\item Powell MJD. The theory of radial basis function approximation in 1990. In Advances in Numerical Analysis, Light FW (ed.), Oxford: Clarendon Press, 1992; 105–203.
		\item J.G. Wang, G.R. Liu. A point interpolation meshless method based on radial basis functions. International Journal for Numerical Methods in Engineering 2002; 54: 1623 – 1648.
		
	\end{enumerate}
\end{frame}

%\section{Thanks}
\begin{frame}	
	%\Huge{\centerline{Thank You!}}
	\begin{block}
		\Huge{\centerline{Thank you}}
		\centerline{shoulinjun@ncic.ac.cn}
	\end{block}
\end{frame}
		
\end{CJK}
\end{document}
